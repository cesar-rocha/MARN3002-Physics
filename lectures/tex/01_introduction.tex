\documentclass[11pt]{article}
%for figures
%\usepackage{graphicx}

%% for graphics this one is also OK:
\usepackage{epsfig}

%% AMS mathsymbols are enabled with
\usepackage{amssymb,amsmath}

%% more options in enumerate
\usepackage{enumerate}

%% links
\usepackage{hyperref}

% for boxing equations, etc
\usepackage{mathtools}
\usepackage{xcolor}
\definecolor{background}{RGB}{204,229,255}
\usepackage{empheq}
\newcommand*\mybox[1]{%
\colorbox{background}{\hspace{1em}#1\hspace{1em}}}
\usepackage{kantlipsum}
\usepackage[fulladjust]{marginnote}
\setlength{\marginparsep}{5mm}
\setlength{\marginparwidth}{2in}
\reversemarginpar

\usepackage{marginnote}

\usepackage{cancel}


%\usepackage{epstopdf}
%\DeclareGraphicsRule{.tif}{png}{.png}{`convert #1 `dirname #1`/`basename #1 .tif`.png}

%% To save typing, create some shortcuts
\newcommand{\ord}{\mbox{ord}}
\newcommand{\Ai}{\mbox{Ai}}
\newcommand{\Bi}{\mbox{Bi}}
\newcommand{\half}{\tfrac{1}{2}}
\newcommand{\p}{\partial}
\newcommand{\com}{\,,}
\newcommand{\per}{\,.}

\newcommand{\defn}{\stackrel{\text{def}}{=}}
\newcommand{\noi}{\noindent}
\def\beq{\begin{equation}}
\def\eeq{\end{equation}}
\def\bseq{\begin{subequations}}
\def\eseq{\end{subequations}}
\def\bal{\begin{align}}
\def\eal{\end{align}}



%% Use Roman font for special numbers and differentials:
\newcommand{\ii}{\mathrm{i}}
\newcommand{\dd}{\mathrm{d}}
\newcommand{\ee}{\mathrm{e}}

%% stop typing all of epsilon and delta
\newcommand{\ep}{\ensuremath {\epsilon}}
\newcommand{\de}{\ensuremath {\delta}}

%% differential operators
\newcommand{\bcdot}{ \boldsymbol{\cdot} }
\newcommand{\cross}{\times}
\newcommand{\grad}{\boldsymbol{\nabla}}
\newcommand{\curl}{\boldsymbol{\nabla} \!\times\!}
\renewcommand{\div}{\boldsymbol{\nabla} \!\bcdot\!}
\newcommand{\lap}{\triangle}
\newcommand{\nabs}{{\boldsymbol{\nabla}}}

\newcommand{\gradh}{\grad_{\!\!h} }
\newcommand{\divh}{\boldsymbol{\nabla}_{\!\!h} \!\bcdot\!}

%% unit vectors
\newcommand{\ihat}{\boldsymbol{i}}
\newcommand{\jhat}{\boldsymbol{j}}
\newcommand{\khat}{{\boldsymbol{k}}}

%% vectors
\newcommand{\uf}{\boldsymbol{u}}
\newcommand{\uvec}{\vec{u}}

%% Abbreviations
\newcommand{\PO}{Physical Oceanography}


\title{MARN3002-Physics: 01 Introduction}
\author{Cesar B Rocha}
%\date{April 5th, 2014}
\date{UConn Avery Point, 2020}

\begin{document}

\maketitle

\section{What is Physical Oceanography?}
Oceanography is the scientific study of the ocean, including its physics,
chemistry, biology and geology.  Oceanographers aim to produce a mechanistic understanding
of the ocean, with its myriad of multidisciplinary processes,
as part of the Earth System. The process-oriented focus and its interdisciplinary nature
are distinguishing characteristics of Oceanography.

The branch of Oceanography that focusses on ocean physics is known as \emph{Physical
Oceanography}. It studies why and how the ocean flows, what sets the distribution
of ocean properties such as temperature and salinity, and how ocean motions
influence (and sometimes control) ocean biogeochemistry and weather and climate.



\begin{figure}[H]\label{wustmap}
\centering
\includegraphics[width=.85\textwidth]{figs/WustSalinityMap.png}
\caption{\small Surface salinity map based on data collected aboard the German ship
R/V Meteor in 1925-1927. Dots represent oceanographic stations, where in-situ measurements and
water samples were collected. Maps like this were the main product of Ocean Geography,
which later evolved into Descriptive \PO. Note the high-salinity tongue in the Northeastern
Atlantic, a fottprint of the salty Mediterranean water that overflows into the Atlantic
through the Strait of Gibraltar. Source: Adapted from Wüst (1978).}
\end{figure}

Initially, \PO{} emerged as a branch of Geography. Its principal products were maps
of the distribution of ocean properties from observations collected in multi-year
transoceanic expeditions, beginning in the 1800s. This oceanic branch of Geography
developed into the subdiscipline \emph{Descriptive \PO{}}, which aims to quatitatively
\emph{describe} ocean motions. Historically, this description was built
with sparse ship-based observations, and more recently, with a combination of
multi-platform observations and realistic computer simulations.

In the early 1900s, \PO{} also began to be viewed as part of
Geophysics, the branch of Physics concerned with Earth phenomena. The field of
\emph{Dynamical \PO{}} emerged, approaching the understanding of ocean flows as a
fundamental problem in Fluid Mechanics and studying it with theoretical and computer
models. A similar historical development occured in
Atmospheric Sciences, with the advent of Dynamical Meteorology in the late 1800s.
In fact, Dynamical Meteorology and Dynamical \PO{} are considered sister sciences,
and most of their topics fall under the umbrella of Geophysical Fluid
Dynamics––the study of fluid flows influenced by rotation and
stratification on Earth as well as on other planets and stars.

\begin{figure}[ht]\label{cm2p6}
\centering
\includegraphics[width=.95\textwidth]{figs/cm2p6.png}
\caption{\small Sea-surface temperature in the Tropical Pacific from a state-of-the-art
realistic ocean simulation. A realistic ocean numerical model is a computer code that solves
the equations of the Dynamical \PO, given initial and boundary conditions provided
by observations. Note the texture of sea-surface temperature, associated with a number of ocean
phenomena. Source: NOAA/GFDL.}
\end{figure}


While today we still talk about observational (descriptive) and theoretical
(dynamical) oceanography, this division has been significantly eroded in the
last couple of decades. The goal of a modern observational study is rarely to only survey a particular region; it generally aims to understand the mechanisms behind the observed phenomena. And dynamical process studies,
both theoretical and numerical, are strongly influenced by observations. Modern
Physical Oceanography focusses on understanding the \emph{physics of ocean processes},
their causes and consequences, using a combination of observations, theoretical
fluid mechanics,  and computer models.

\begin{figure}[ht]
\centering
\includegraphics[width=.95\textwidth]{figs/TimeSpaceScales.png}
\caption{\small Diagram of space and time scales of a number of ocean phenomena.
In general, larger spatial
scales are associated with longer time scales. Source: Talley et al., \textit{Descriptive Physical Oceanography}.}
\label{scales}
\end{figure}

Ocean physical processes can be local or regional or global, small or large, fast
or slow. Figure \ref{scales} shows a space-time diagram of ocean phenomena, including
bubbles, surface-gravity waves, tides, tsunamis, and global overturning
(thermohaline) circulation. The range of scales of ocean variability is extraordinary,
varying from from miliseconds and milimiters to thousands of years and tens of thousands of
kilometers. Undersdanting these myriad phenomena and modeling their interactions is
the main goal of Physical Oceanography.

In this module of MARN3002, \emph{we will study the basic physics of some of these phenomena
} from observations and theory. I will emphasize \emph{major concepts and ideas} of \PO. Your
expected learning outcome is a newfound \emph{physical intuition} for how ocean motions work. Such an intution is critical to learning the details of \PO{} and its relationship to other areas of Oceanography, including the topics explored in the biology module of this class. While Calculus is a pre-requisite for this class, I will strive to keep Math at a moderate level, focussing on what the relevant equations mean (what physics they contain) rather than deriving and solving them.


\section{The physical/geographical setting}


We begin by describing the container where ocean motions occur---planet Earth
and its oceanic basins.

\subsection{Spherical Earth}

\begin{figure}[ht]
\centering
\includegraphics[width=.55\textwidth]{figs/Sphere.png}
\caption{\small A spheroid representing Earth, with coordinates latitude ($\theta$),
longitude ($\lambda$) and radius $r$. The radius of the sphere (the distance from the center
to any point on the sphere's surface) is the radius $R$. Source: Adapted from geokov.com.}
\label{EarthSphere}
\end{figure}

 Earth is an oblate spheroid, a sphere slitghtly
flattened at the poles. This flattening is due to Earth's rotation around its vertical
axis, and it implies
that the polar radius is a bit (0.3\,\%) smaller than the equatorial radius. The rate
of Earth's rotation is
\begin{align}
% \marginnote{Recall that\\ 24\,h = 86400\,sec}[1cm]
|\vec{\Omega}| &= \frac{2\pi}{1\, \text{day}} + \frac{2\pi}{365\, \text{days}}\nonumber\\
               &= \frac{2\pi}{86400\, \text{sec}} + \frac{2\pi}{365 \times 86400 \, \text{sec}}
               \approx 7.29\, 10^{-5}\, \text{rad}/\text{sec}\per
\end{align}
The direction of the rotation vector $\vec{\Omega}$ is perpendicular to the equator, pointing north (see figure \ref{EarthSphere}).

It is natural to use a spherical coodinate system to describe positions and displacements on Earth.
This consists of the distance from the center of the sphere ($r$) plus two angles $\theta$
and $\lambda$.  Earth Scientists adopt the convention of measuring $\theta$ relative to the
equator (the circle in the direction perpendicular to the rotation axis that splits Earth in half). The \emph{latitude} $\theta$ varies from $-90^{\circ}$ (South Pole) to
$+90^{\circ}$ (North Pole). For mathematical calculations, we almost \emph{alway} measure angles
in radians (rad). Recall that $360^\circ = 2\pi\,\text{rad}$, so
\beq
\text{angle in radians} =  \frac{\pi}{180} \times \text{angle in degrees} \per
\eeq
Thus, in radians, $\theta$ ranges from $-\pi/2$ to $\pi/2$. Longitude $\lambda$ goes
around the globe, varying from 0$^\circ$ (0) to 360$^\circ$ (2$\pi$). The reference longitude
passes through the Greenwich, UK. An equivalent longitude range is from -180$^\circ$ ($-\pi$) to
+180$^\circ$ ($+\pi$), with negative angles west of Greenwhich. The curve along
constant longitudes are called \emph{meridians}. We refer to longitudinal direction as
the \emph{meridional direction}. The latitudinal circles are called {parallels}, and we refer to
the direction along parallels as the \emph{zonal direction}.

The average radius of Earth is $R = 6371$\,km. Hence one degree of latitude is
equivalent to:
\beq
1^\circ\, \text{latitude} = \frac{2\pi R}{360} = 111.20\,\text{km}\per
\eeq
Things are slightly different for longitude. The circumference
around Earth at the constant latitude $\theta$ has a radius $R_\theta = R \cos\theta$. Thus
\beq
1^\circ\, \text{longitude} = \frac{2\pi R}{360}\cos\theta\per
\eeq
Note that  $1^\circ\, \text{longitude}
\le 111.20$\,km, with the equality holding at the equator, because
$\cos 0 = 1$. The meridians converge poleward, with the distance between them
diminishing with increasing latitude. For instance, at the latitude of Avery Point $\theta = 41.3179^\circ$N, $1^\circ\, \text{longitude} = 83.52$\,km. Further poleward, at $\theta = 85^\circ$,
 $1^\circ\, \text{longitude} = 9.67$\,km.

Distances in the ocean are also reported in \emph{nautical miles} (Nm). One Nm is equivalent
to one minute (i.e., one-sixtieth) of latitude:
\beq
1\,\text{Nm} = \frac{1^\circ}{60} = \frac{111.20}{60} \,\text{km} = 1.853\,\text{km}\per
\eeq

\subsection{Ocean topography}
Figure \ref{ETOPO} shows a map of the sea-floor bathymetry (bottom topography)
on the global ocean. The values are negative, representing the distance from the sea surface to the
ocean bottom. These data are from the ETOPO5 dataset, compiled by NOAA's National Center for Environmental Information; the resolution is 5 minutes of latitude (about 9\,km).

\begin{figure}[ht]
\centering
\includegraphics[width=.85\textwidth]{figs/TopographyWithLine.png}\\
\caption{\small World map of ocean topography and the location of a transatlantic bathymetric section along 41.3179$^\circ$\,N, the latitude of Avery Point. Plotted with ETOPO5 dataset. }
\label{ETOPO}
\end{figure}

Moving seaward from the coast lies the continental margin, an extension of the continent.
The continental margin has three main components: shelf, slope and rise (or elevation). These differentiated primarily by
their inclination, the topographic gradient. The continental shelf starts at the coast and is the
flattest region of the continental margin. On the shelf, the bottom depth increases by about 1 m
every 500\,m. In other words, the topographic gradient is 1:500. Continental shelves vary in extension, from 10\,km to 200\,km, width an average with
of 70\,km. At the seaward edge of the shelf is the shelf break, where the topographic gradient increases
significantly to 1:20. The average depth at the shelf break is 130\,m. The slope is a narrow and steep region of the continental margin, where the bottom depth rapidly increases seaward. At about 3000-4000\,m, the topographic gradient becomes gentler, given way to the continental elevation or rise, separating the
continental margin from the abyss. By and large, the ocean abyss has rough topography, with oceanic mountain chains (such as mid-ocean ridges), isolated seamounts and trenches. The average depth of the ocean is 3734\,m. The deepest documented region of the world ocean is the Mariana trench,
in the Northwest Pacific, where depths reach more than 11000\,m (11\, km).

\begin{figure}[ht]
\centering
\includegraphics[width=.85\textwidth]{figs/ETOPO5CDF.png}
\caption{\small Cumulative probability function (CDF) of ocean topography, based on the ETOPO5 dataset.}
\label{ETOPOCDF}
\end{figure}

Figure \ref{ETOPOCDF} shows the cumulative probability distribution (CTD) the ocean
topography in figure \ref{ETOPO}. In essence, a CDF is a fractional histrogram, which
meaures the cumulative probability of the data.  In figure \ref{ETOPOCDF}, the cumulative
probability starts at the deepest levels, moving to the right. The maximum probability is
1 (100\,\%), stating the obvious fact that 100\,\% of the ocean has depths below 0\,m. The topography CDF
shows that continental shelves (roughly regions shallower than 200\,m) occupy less than 15\,\% of
the world ocean, and the continental slope accounts for about 5\,\% the total area. Most of the ocean
is deeper than 2000\,\%, although it is rarely deeper than 6000\,m (\<0.05\,\%).


\subsection{The aspect ratio}

\begin{figure}[ht]
\centering
\includegraphics[width=.99\textwidth]{figs/TopographySection.png}
\caption{\small A transatlantic bathymetric section along 41.3179$^\circ$\,N, the latitude of Avery Point, indicated in figure \ref{ETOPO}. Plotted with ETOPO5 dataset. }
\label{ETOPOsec}
\end{figure}


While most of the ocean is deep for our human scale, oceanic basins are extremelly shallow
on a planetary scale. Even the deepest point in the Mariana Trench is shalloq in
comparison with Earth's radius or circumference. The ocean is a \textit{thin shell} in
outter 0.08\,\% of Earth. In fact, when we plot long vertical sections of ocean topography,
as in Figure \ref{ETOPOsec}, the vertical coordinate needs to be grossely exageraged in order
to render a detailed view of topographic features. From figure \ref{ETOPOsec}, the North Atlantic
basin spans 60$^\circ$ of longitude at 41.3179$^\circ$\,N. Thus, the North Atlantic width is
\beq
60^{\circ}\,\text{longitude} \times \frac{111.20\,\text{km} \times \cos 41.3179^\circ}{1^\circ\,\text{longitude}} \approx 5011\, \text{km}\per
\eeq
The average depth of the North Atlantic at this latitude is about 3000\,m. To estimate
the relative size of the North Atlantic depth to its width, we calculate
their ratio using the same \emph{units}:
\beq
\frac{\text{depth}}{\text{width}} = \frac{3000\,\text{\cancel{m}}}{5011 10^3\,\cancel{\text{m}}}
= 0.0006\per
\eeq
In other words, the North Atlantic thickness is about 0.06\,\% of its width. The
relative size of thickness to width is called \textbf{aspect ratio}. The aspect
ratio is a \emph{non-dimensional number}. For most practical purposes, we are more
interested in the orders of magnitude. We thus
define the aspect ratio as
\begin{empheq}[box=\mybox]{align}\label{aspectratio}
\alpha \defn \frac{H}{L} \com
\end{empheq}
where $H$ the order of magnitude of thickness (or vertical length scale) and $L$ is the
order of magnitude of width (or horizontal length scale). For the
North Atlantic, $H \sim 10^3\,\text{m}$ and $L \sim 10^6\,\text{m}$, and so $\alpha = 10^{-3}$.

As we will see, this tiny aspect ratio of the ocean is a powerful geometrical constraint on the
ocean circulation, because it implies that many circulation features are much wider than deep.
For example, the Gulf Stream---the warm current that flows poleward off the US East coast--is about 100\,km and 1000\,m deep. The aspect ratio of the current is $10^{-2}$. As we will see, small aspect ratio flows have tiny vertical velocities compared to horizontal velocities. To an excellent approximation,
the Gulf Stream flow is horizontal.

\subsection{The local plane approximation}
Because the ocean is a thin spherical shell, we may employ a plane approximation to
avoid spherical coordinates when studying motions on regional and local domains. For this
approximation to be valid, the lateral scale of the domain must much smaller than Earth's radius:
\beq
\frac{L}{R} \ll 1\per
\eeq
In the cartesian system of the local plane, we generally align the x-axis with the zonal direction
and the y-axis with the meridional direction. Sometimes, there exists a better coordinate system
for a particular domain. For example, for studying motions in the Long Island Sound, it is convinient to align the x-axis in the along-sound direction and the y-axis in the across-sound direction.

\end{document}

In \PO{}, we use \emph{non-dimensional numbers}, such as the ration above, as a means
to compare different terms

\sim \frac{10^3}{10^6} = 10^{-3} \ll 1

\begin{empheq}[box=\mybox]{align}\label{geobarvec}
f_0 \uvec_h =  g \, \khat\cross\gradh \eta\com
\end{empheq}
o
